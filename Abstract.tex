The ability to learn is often a desirable property of intelligent systems which can make them more adaptive.
However, it is difficult to develop sophisticated learning algorithms that are effective.
One approach to the development of learning algorithms is to evolve them using evolutionary algorithms.
The evolution of learning is interesting as a practical matter because harnessing it may allow us to develop better artificial intelligence;
it is interesting also from a more theoretical perspective of understanding how the sophisticated learning seen in nature -- including that of humans -- could have arisen.
A potential obstacle to the evolution of learning when alternative behavioral strategies (e.g., instincts) can evolve is that learning individuals tend to exhibit ineffective behavior before effective behavior is learned.
Nurturing, defined as one individual investing in the development of another individual with which it has an ongoing relationship, is often seen in nature in species that exhibit sophisticated learning behavior.
It is hypothesized that nurturing may be able to increase the competitiveness of learning in an evolutionary environment by ameliorating the consequences of incorrect initial behavior.
The approach taken is to expand upon a foundational work in the evolution of learning to enable also the evolution of instincts and then examining the strategies evolved with and without a nurturing condition where individuals are not penalized for mistakes made during a learning period.
It is found that learning is more likely to evolve than instincts in the presence of nurturing than in the absence of nurturing.